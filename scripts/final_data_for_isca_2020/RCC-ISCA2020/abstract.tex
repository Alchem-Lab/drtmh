\begin{abstract}

On-line transaction processing (OLTP) applications
require efficient distributed transaction 
execution. When a transaction accesses 
multiple records in remote machines, 
network performance is a crucial factor 
affecting transaction latency and throughput.
Due to its high bandwidth and very low
latency, RDMA (Remote Direct Memory Access) has achieved much higher performance 
for distributed transactions than traditional TCP-based systems. 
RDMA provides primitives for 
both two-sided and one-sided communication.
Although recent works have intensively studied
the benefits of RDMA in
distributed transaction systems,
they either focus on
primitive-level comparison of two 
communication models (one-sided vs. two-sided) or 
only study one concurrency control protocol.
Comprehensive understanding of the implication
of RDMA for various concurrency control protocols
is an open problem. 

In this paper, we build {\em \projectname}, the first
unified and comprehensive RDMA-enabled distributed transaction processing framework supporting 
six concurrency control protocols 
using either two-sided or one-sided primitives.
We intensively optimize the performance
of each protocol without bias,
using known techniques such as co-routines, outstanding requests, and
doorbell batching.
Based on \projectname, we conduct the first
and most comprehensive
(to the best of our knowledge) study
of the six representative distributed
concurrency control protocols 
on two clusters with different 
RDMA network capabilities.





%Distributed database management systems (DBMS) have long been using concurrency control protocols like 2-phase locking (2PL)~\cite{Bernstein:1981:CCD:356842.356846}, multi-version CC (MVCC)~\cite{bernstein1983multiversion} and Optimistic Concurrency Control (OCC)~\cite{kung1981optimistic} to ensure atomicity and serializability of distributed transactions. Other concurrency control protocols like CALVIN\cite{Thomson:2012:CFD:2213836.2213838}, 
% MaaT~\cite{mahmoud2014maat} 
%and Sundial~\cite{yu2018sundial} have also been proposed in recent years. Meanwhile, High-speed network technology like Remote Direct Memory Access (RDMA) is getting popular and used by recent transaction processing systems like DrTM~\cite{wei2015fast} and RTX~\cite{wei2018deconstructing} due to its low latency. Unfortunately, it is hard to compare and evaluate concurrency control protocols under the context of using RDMA. We developed RCC, a framework containing various representative build-in concurrency control protocols with RDMA support. we provide detailed designs and evaluations of each protocol in terms of both using RDMA-enabled RPC and using one-sided RDMA primitives. As far as we know, this is the first framework where researchers can compare different concurrency control protocols on one single system under the context of RDMA.

\end{abstract}