%\vspace{-2mm}
\section{Related Work}
%\vspace{-2mm}
{\bf Comparisons among concurrency control protocols}
%Prior work has made valuable comparisons on different concurrency control protocols. 
\cite{agrawal1987concurrency} uses modeling techniques to reveal the hidden connections between protocols' underlying assumptions and their seemingly contradictory performance results.\cite{huang1991experimental} compares three concurrency control protocols in real-time database systems but only restraints to optimistic ones. \cite{yu2015evaluation,yu2014staring} focuses on the scalability issues and examines seven concurrency control protocols on a main-memory DBMS on top of a simulated 1024-core system.  Deneva~\cite{harding2017evaluation} is the recent work comparing distributed concurrency control protocols in a single unified framework. 
%Deneva executes all transactions as stored procedures that run on servers, and analyzes the performance of different protocols under different settings like network speed, node number, update rate and contention. 
\projectname 
takes the first step in comparing different protocols under the context of various RDMA primitives.

{\bf Comparisons between RDMA primitives}
%Before \projectname, there has already been work comparing RDMA primitives. 
\cite{kaminsky2014using} compares the use of RDMA \texttt{WRITE} and RDMA \texttt{READ} when constructing a high performance key-value system. \cite{dragojevic2014farm} finds out that RDMA \texttt{WRITE}'s polling significantly outperforms \texttt{SEND} and \texttt{RECV} verbs when constructing the FaRM's communication subsystem. \cite{kalia2016fasst} shows that UD-based RPC using \texttt{SEND} and \texttt{RECV} outperforms one-sided primitives. \cite{wei2018deconstructing} did more primitive-level comparisons with different payload size. Compared
to them, \projectname compare the primitives
with a much wider range of concurrency control algorithms
on two clusters with different RDMA capabilities.

%applicRDMA nation level in order to reach useful insights in constructing concurrency control protocols.

{\bf Distributed transaction systems}
High performance transaction systems have been investigated intensively~\cite{Thomson:2012:CFD:2213836.2213838,corbett2013spanner,tu2013speedy,dragojevic2015no,chen2016fast,wei2018deconstructing,lee2015implementing}. Most of them focus on distributed transaction systems~\cite{corbett2013spanner,dragojevic2015no,chen2016fast,wei2018deconstructing} since it is more challenging to implement a high performance transaction system with data partitioned across the nodes. 
Some works, e.g., \cite{lee2015implementing,dragojevic2015no,chen2016fast,wei2018deconstructing,kalia2016fasst}, focus only on one protocol (i.e., some variants of \occ). Other works like~\cite{wei2015fast,yu2018sundial,Thomson:2012:CFD:2213836.2213838} explore novel techniques like determinism or leasing. However, these works did not explore the opportunity of using RDMA networks.